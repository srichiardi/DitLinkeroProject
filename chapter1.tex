\chapter{Introduction}

\section{Project background}
The European Union Intellectual Property Office, in coolaboration with the
OECD, estimated that in 2013 the value of imported counterfeit good was
\euro338 billion, which corresponded to 2.5\% of the total imports in the world
trade \cite{OE16}. The report also calculates that 5\% of imported good in EU
are counterfeit worth up to \euro85 billion. Counterfeit is an unfair commercial
practice which takes advantage of the brand identity and presence on the market built by the brand
owner, without incurring in the same costs of brand development (product design,
quality standards and marketing). Other than a financial damage for the original
brands, counterfeit products may pose safety risks for end consumers as much as
for the people that work to produce them, since they evade the strict quality
and safety standards set by national and international agencies around the
world. From counterfeit iphone batteries that explode to counterfeit air-bags
that do not trigger, they all can cause physical damage to consumers.

E-commerce has been rapidly expanding since it's early days in mid '90 when
online marketplaces like eBay and Amazon were launched. Although it is hard to
make a global estimate of market share for e-commerce companies, if we take the
US market, the Census Bureau estimated that in 2009 online sales accounted for
4\% of all retail sales, whereas in 2017 the e-commerce market share went up to
9\% \cite{USDC18}. As legitimate retailers increase their presence online,
shops selling counterfeit follow suit.

It is not surprising that many companies started to invest in fighting this
trend, and that a good portion of the battleground is online. This tool is
designed to help users to extract data related to a specific brands from online
platforms such as eBay, Mercadolibre, Allegro, and present it in a tabular
format.

Online marketplaces offer API open to developers in order to enable automated
interaction with their platforms. This tool will focus on extracting sales and
business registration data. The data will be stored in a database so that users
can keep historical records of all their queries. At the same time, users will
be able to leverage the growing dataset to link new investigations to old cases
across all platforms whose data has already been stored in the database. For
instance, serching the email address of a shop that deals counterfeit Diesel
jeans may return details of multiple businesses registered on eBay and
Mercadolibre at different times, as well as information about the administrator
of a facebook page about counterfeit Raiban glasses. Users can use this tool to
estimate brand damage caused based on the volume of sales, as well as a
forensics platform to facilitate identity attribution of potential
counterfeiters.

\section{Background}
Identifying online counterfeit items starts with reasearches of a brand or
product online presence. The aim is to identify sellers that offer a branded
product sold at a price point below average retail price. An investigator would
ideally prioritize sellers with higher business turnover, and ideally located in
jurisdictions where legal action is a dependable and impartial option. For
instance countries like Russia and China, to name few, do not always offer
adequate protection for European and North American companies, making it more difficul to pursue
compensation from actors located withing their borders.

There are already different online companies that offer brand monitoring
services: they range from keyword web-crawlers, to consumer sentiment analytics
based on social platforms, to anti-counterfeit detection. We will review those
companies and their services in details in the next chapter.

GDPR is the latest Europen Union legislation about processing of personal
identifiable, privacy protected data belonging to european citizens. GDPR
affects the implementation of this project in two ways: first, and most obvious,
because users will submit limited personal data in order to be able to access
Linkero services. Something as simple as an email address required to get
notification for password recovery or data collection completed confirmation, is
considered a PII (personal identifiable information) and thus is protected under
GDPR. But GDPR affects also the collection of public personal details available
online. This is the compliance requirement that is specific to companies
providing \emph{web scraping} services. When we deal with open source
intelligence and attribution, PII are the most valuable data for
anti-counterfeit investigations, therefore we will talk about how GDPR affects
services like Linkero and what are the requirements to guarantee compliance.


\section{Objectives}
The objectives set for this project fall in two main categories: business
and personal.

In relation to the first type, the goal is to build a business-ready web
platform for brand-monitoring investigations. Being \emph{business ready}
means presenting a final application that runs on a virtual server directly
connected to the public internet; that implements all industry starndard
security features to guarantee that only authorized users can access its
features and data; and is scalable, designed to accomodate an exponentially
gowing number of users. The other aspect of the first goal refers to the ability to
provide basic case management, data extraction and keyword searches capabilities
tailerd for brand monitoring and anti-counterfeit investigations.

On a personal level, I aim at gaining a hands-on understanding of specific web
technologies (e.g. JQuery, Django framework, NGINX, uWsgi), NoSQL databases
(MongoDB, Redis), ansychronous and non-blocking programming techniques
(multi-threading, AJAX, Pjthon Celery), development methodologies and testing
practices.


\section{Scope}
This project will consist of a ready to deploy and use system for case
management and web scraping data using eBay public API.

Regular users will be able to login, change password, set their preferred email
address for general notifications and report delivery, launch queries, download
data from completed reports, delete old queries, and perform keyword searches
within all data collected and stored by any user in the database.
Also the system will be configured to provide security protection against external
unauthorized access and use of the system. Site administrators will manage the
registration, password reset and deletion of users profiles.


\section{Challenges and learning requirements}
There are technical and personal challenges that have some kind of impact
on the delivery of this project. To start with I am new to most of the
technology used: from the basic configuration and administration of a Unix
server, NoSql databases, client side web technology, SSL and DKIM protocols,
etc\ldots. This means that part of the time dedicated to the project will be
spent bringing my knowledge and familiarity to a level that suits the project's
requirements. Another element of technical challenge is determined by the use of
a virtual server with only 1GB of memory and 1 single CPU. These limits are
mainly dictated by financial reasons: this is the cheapest available VPS, which
comes at \$5.00 USD a month. Considering that I needed a VPS for the entire
development and testing of the project, which took almost 2 years, anything more
expensive would have impancted my family finances. The other problem is that
limited memory and processing power started to show their impact and the code
had to optimized accordingly.

On a presonal level I will have to take into account all the extra pressure
that comes from being a father in a young family and professional
responsibilities of a full time job.


\section{Deliverables}
The completed project will include:

\begin{itemize}
  \item full documentation of the system design, development and functions;
  \item a working implementation of the system;
  \item a PowerPoint presentation of the project.
\end{itemize}

