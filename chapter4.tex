\chapter{Testing and deployment}

\section{Testing strategy}
Testing a system has two meaning: we need to find out if the system build works
as intended (verification testing), but we also want to know if we built the
right system (validation testing) \cite{VV08}. For this project I will assume
I did build the right system, and I will focus on verifying that what I built
performs correctly and complies with its original requirements.

Testing is a tricky business, since its ability to detect issues is only as
good as the set of test cases that are developed for this purpose. Testing
exhaustively every outcome of a system component is not a feasible solution,
since even a very simple program consisting of a single \texttt{for} loop
running 100 iterations, testing a condition and returning a binary output, could
generate 2\textsuperscript{100} different outcomes, which would mean that a
brute-force testing with a machine able to execute 10 million instructions per
second, would take $3 \times 10\textsuperscript{14}$ years to complete
\cite{VV08}.

The alternative is to chose a limited set of test cases. The question then
becomes, how do we chose an \emph{adequate} set of test cases? The answer is a
matter of priority based on the original agreement between the final user and the
system developer:
\begin{itemize}
  \item the testing could prioritize making sure that every requirement is met
  precisely as defined, which is considered \textbf{coverage-based testing}, but
  this approach assumes that requirements are defined in quantitative terms;
  \item or testing could focus on fault detection, \textbf{fault-based testing},
  which means that a lot of effort is first put into creating a testing
  mechanism which needs to be calibrated by seeding intentional faults within
  the system so that we can calculate the test's detection rate;
  \item finally, test cases can be designed based on known errors
  that are normally found in any system, such as \emph{off-by-one} errors. The
  testing therefore will focus con exploring the boundary points \cite{VV08}.
\end{itemize}

The above classification of test techniques applies to the component-level, so
individual class and object methods are tested independently from the rest of
the system, this practice is called \textbf{unit-testing}. Linkero was developed
following the Extreme Programming methodology, which requires component tests to
be written before the module to test is actually written. It is a test-driven
approach that forces the developer do think in advance and anticipate the
potential cases of failure as much as the expected outcome \cite{VV08}.

In a traditional testing approach, unit-testing is followed by
\textbf{integration-testing}, when individual components are then assembled and
tested during their interaction with one another. Eduard Berard points out that
integration testing is a practice that originates from a procedural software
development culture and does not apply as well to object oriented languages. A
solution is \emph{use-based} testing, which proceeds by first testing classes
that are generally independent, and then proceeding by including dependent
classes and continuing to the next layer \cite{EB93}. Linkero, being based on
Python, is built following the object-oriented style, and does not displays that
many layers, therefore integration testing will not be performed.

\emph{System testing} is the last stage of testing before deployment, in which
the system is tested as a whole. There are different type of system testing:
\begin{description}
\item[Recovery testing] is designed to cause failure status to verify how the
system is able to restore its original state.
\item[Security testing] verifies that protection mechanisms are in place against
the most common attacks.
\item[Stress testing] is a test that executes instructions that demand a very
high level of resources in order to assess the system's ability to sustain a
significant workload \cite{RP05}.
\end{description}

In this project I will consider only security testing, which is one of the main
requirements identified on the outset.


\section{Unit testing}
To guarantee the correct functioning of the system, I decided to test each
method of the core business logic, which is the part of the system that takes
users' report requests and retrieves data from external sources.

Python has its own official testing framework: the \texttt{unittest} module.
This framework allows to write test cases and, obviously enough, test if the output
return as expected or not. Django integrates \texttt{unittest} in its libraries,
and provides by default a \texttt{test.py} file for each web-app created, so
that projects' tests can be stored and reused repeatedly.

This is an example of the approach that I took for this project. The test
is written before creating a method that would search for eBay items based on
keywords within list of regional eBay sites (the type of results of an items
search on eBay are dependant on the region selected: ``skateboards'' on ebay.com
will return different results compared to ebay.co.uk). In this test case I
wanted to check the ability of the \texttt{find\_items()} method to return a
meaningful error message with the wrong eBay site was provided. I prepared a
test in \texttt{test.py}, feeding a wrong eBay site to the
\texttt{find\_items()} method of the \texttt{EbayApi} class:
\begin{lstlisting}[language=Python, breaklines=true]
from django.test import TestCase
from cases.ebayapi import EbayApi, EbayBadSite

# Create your tests here.
class FindItemsTest(TestCase):
    def setUp(self):
        self.ea = EbayApi()
    
    # test if the method accepts an eBay site not correctly formatted
    def test_bad_ebay_site(self):
        wrong_site = 'PK'
        with self.assertRaises(EbayBadSite):
            self.ea.find_items(ebay_site=wrong_site, keywords="baby oil")
\end{lstlisting}

The first test fails as expected, because the exception class
\texttt{EbayBadSite} is not defined:
\begin{lstlisting}[language=bash, breaklines=true]
(Projects) stefano@attila:~/Projects/linkero$ python manage.py test
Creating test database for alias 'default'...
System check identified no issues (0 silenced).
E
======================================================================
ERROR: test_bad_ebay_site (cases.tests.FindItemsTest)
----------------------------------------------------------------------
Traceback (most recent call last):]
  File "/home/stefano/Projects/linkero/cases/tests.py", line 12, in test_bad_ebay_site
    with self.assertRaises(EbayBadSite):
NameError: name 'EbayBadSite' is not defined

----------------------------------------------------------------------
Ran 1 test in 0.004s

FAILED (errors=1)
Destroying test database for alias 'default'...
\end{lstlisting}

The second iteration brings a slightly different result. The exception class is
now available, but there is no code that raises it, instead a different error
arises when the bad input is used to retrieve any of the pre-loaded sites, the
dictionary object throws a \texttt{KeyError} exception, signifying that the
bad input \emph{PK} is nowhere in the list of pre-defined country codes:
\begin{lstlisting}[language=bash, breaklines=true]
Creating test database for alias 'default'...
System check identified no issues (0 silenced).
E
======================================================================
ERROR: test_bad_ebay_site (cases.tests.FindItemsTest)
----------------------------------------------------------------------
Traceback (most recent call last):
  File "/home/stefano/Projects/linkero/cases/tests.py", line 13, in test_bad_ebay_site
    self.ea.find_items(ebay_site=wrong_site, keywords="baby oil")
  File "/home/stefano/Projects/linkero/cases/ebayapi.py", line 39, in find_items
    'GLOBAL-ID' : globalSiteMap[ebay_site]['globalID'],
KeyError: 'PK'

----------------------------------------------------------------------
Ran 1 test in 0.004s

FAILED (errors=1)
Destroying test database for alias 'default'...
\end{lstlisting}

Eventually, some iterations later, after including an initial step to check the
input validity \ldots
\begin{lstlisting}[language=Python, breaklines=true]
    # check input provided
    try:
        global_site_id = globalSiteMap[ebay_site]['globalID']
    except KeyError:
        raise EbayBadSite("Wrong ebay site input!")
\end{lstlisting}

\ldots the test is a pass:
\begin{lstlisting}[language=bash, breaklines=true]
Creating test database for alias 'default'...
System check identified no issues (0 silenced).
.
----------------------------------------------------------------------
Ran 1 test in 0.003s

OK
Destroying test database for alias 'default'...
\end{lstlisting}

This iterative approach was repeated for the other classes and methods. The
class \texttt{EbayApi} for instance was tested for the following cases:
\begin{itemize}
  \item test if the \texttt{EbayBadSite} error raised by the
  \texttt{get\_multiple\_items()} method when a bad eBay site is passed as
  argument;
  \item test if the \texttt{EbayTooManyItems} error is raised by
  \texttt{get\_multiple\_items()} method if more than 20 items are passed at a
  time.
  \item test if the \texttt{EbayInvalidSellerId} error is raised by the
  \texttt{get\_seller\_details()} method if an non-existent seller id is passed as
  argument.
\end{itemize}

The class \texttt{CasesView} was tested for:
\begin{itemize}
  \item the \texttt{post()} method returning a JSON response equivalent to
  \begin{verbatim}
  { 'status' : 'fail',
  'error_msg' : 'provide at least one seller id or keyword'} \end{verbatim} when
  the user leave black both seller id and keyword fields.
  \item the \texttt{get()} method returning a JSON response equivalent to
  \begin{verbatim}
  { 'status' : 'fail',
  'error_msg' : 'invalid input provided'} \end{verbatim} when the wrong platform
  input is provided to search for cases.
\end{itemize}


\section{Security testing}
Security testing was done with Zed Attack Proxy (ZAP), an open source web
application security scanner. The application is developed maintained by the
Open Web Application Security Project, an online community that focus on online
security awareness and support for security testers and penetration testers
\cite{zap}, and it is considered one of the most reliable to detect
web-based vulnerabilities.

I performed the default active scan against \url{https://linkero.ie}. This
first attack was performed on the public pages \texttt{/login/} and
\texttt{/logout/}, and the tool identified six low priority issues:
\begin{description}
  \item[Password autocomple enabled]: this option in any other input field
  allows the browser to predict the value while the user is typing. Values are
  predicted based on previous inputs, previously stored. Allowing the password
  to be stored for future auto-completion, is not a good idea. The fix involves
  adding the option \texttt{autocomplete=off} in the HTML password input field.
  \item[Web Browser XSS Protection Not Enabled]: X-XSS-Protection is a
  value set by the web server in the HTTP response header to allow modern
  browsers to enable XSS protection feature. This setting is disabled in default
  Django settings. To fix this is necessary to specify
  \texttt{SECURE\_BROWSER\_XSS\_FILTER = True} in the \texttt{settings.py} file.
  \item[X-Content-Type-Options Header Missing]: another response header value,
  this one clarifies if the client browser is allowed to sniff MIME content or
  should only render the content based on the content-type declared. This option
  was introduced to prevent MIME abuse attacks, where malicious content would be
  uploaded on a site, and tricking the site visitors to render it in a different
  way from the official content type of the website, and potentially executing
  malicious code. Django keeps this option disabled, and considering that
  Linkero does not allow users to upload any content, it represents a very low
  risk. In any case, this can be fixed by declaring
  \texttt{SECURE\_CONTENT\_TYPE\_NOSNIFF = True} in the \texttt{settings.py}
  file.
  \item[Incomplete or No Cache-control and Pragma HTTP Header Set]: these
  response header values prevent the client browser from caching the page
  content. In this case the fix relies on setting the \texttt{@never-cache}
  decorator on the web-app page that I do not want the user to cache. This has
  to be done on every page.
  \item[Cookie Without Secure Flag]: session cookies make sure that the user
  that is contacting the server is the same that originally authenticated during
  a login event, and was not replaced by another malicious actor after
  authentication. This is only possible if sessions are not intercepted and read
  during man-in-the-middle attacks. Django does not set the \texttt{Secure} flag
  by default because not every site need user authentication and not every site has
  SSL enabled. To fix this issue is sufficient to declare
  \texttt{SESSION\_COOKIE\_SECURE = True} in \texttt{settings.py} file.
  \item[Cookie No HttpOnly Flag]: finally, this header controls cookie access by
  JavaScript. In a public forum, where un-vetted code can be uploaded by anyone,
  a malicious actor could upload some JavaScript code designed to read other
  clients' cookies, stealing their sessions. Linkero restricts significantly the
  users ability to upload code, and there are limited options to share
  user-generated content. On the other end, part of Linkero's client JQuery
  script needs to access clients' cookies to append CSRF tokens to AJAX
  requests. I decided to keep this option disabled, considering the risk near to
  null and the benefits of improved usability that AJAX requests provide.
\end{description}

A second scan, after the necessary fixes were put in place, returned only a
\textbf{Cookie No HttpOnly Flag} issue, which is something I expected and
decided to accept.


\section{Deployment}
The deployment is relatively simple, since the platform was developed on the same
virtual server that will be employed in production mode.

The Django application framework can run in \emph{developer} or in
\emph{production} mode. The first mode is designed to return detailed traceback
reports, whenever the application generates an error. This is very useful during
development stage, but it is a clear security concern if left enabled, since
error messages are public.

The other change required before deployment relates to the passwords that were
in use during the testing and development stages, especially if hard-encoded in
the \texttt{settings.py} file. The entire Django application and its settings were
cyclically uploaded on GitHub\texttrademark. I do not have a private
GitHub account, therefore every information, including passwords used during
development are publicly visible at \url{https://github.com/srichiardi/linkero}.
That means that the passwords to access MariaDB and MongoDB need to be changed.
Django also provides a default \texttt{SECRET\_KEY}, a 50 chars long
alphanumeric string used for:
\begin{itemize}
  \item generate unique ``recover my account'' URLs, or other one-time access
  pages;
  \item ensure that data in hidden form fields has not been tampered with
  \cite{djsecr}.
\end{itemize}
The \texttt{SECRET\_KEY} was stored just like any other password in the
\texttt{settings.py} file, as such it needs to be replaced by a new one
by editing directly the file stored on the actual virtual machine serving
the site.
