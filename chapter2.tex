\chapter{Literature review}

\section{Introduction}
This chapter will provide an overview of different topics that informed design
decisions made during the initial stages of the project. We will start looking
at the discipline of Open Source Intelligence, which refers to all those
protocols and techniques used by government and private agencies to piece
together intelligence reports using publicly available sources, since Linkero is
a tool that facilitates the structured collection and analysis of a limited
portion of open source data. The legal aspect of data collection and analysis
will be explored with a reasoned summary of what is GDPR and how it impacts
similar online services. Three sections will be dedicated to industry level best
practices in relation to the security of servers facing the public internet, the
usability of software interfaces and current guidelines to build reusable,
mantainable and expandable code. Finally we will review briefly what are some of
the current services already offering brand monitoring tools and how they differ
from one another.


\section{Open Source Intelligence}
Open Source Intelligence (more commonly referred to as OSINT) is a relatively
young discipline, that is concerne with the art of piecing together strategic
intelligence from publich sources of information. Michael Bazzel, a leading OSINT expert,
defines it as:
\begin{displayquote}
any intelligence produced from publicly available information that is collected,
explited, and disseminated in a timely manner to an appropriate audience for the
purpose of addressing a specific intelligence requirement. for the CIA, it may
mean information obteined from foreign news broadcasts. For an attorney, it may
mean data obtained from official government documents that are available to the
public. For most people, it is publicly available content obtained from the
internet \cite{MB15}.
\end{displayquote}

As Bezzell explains, OSINT is not necessarialy based on online sources, at least
in its most broad definition. Journalistic style, old fashion dossiers filled
with newspaper clippings are a form of OSINT. However it is fair to say that
whenever OSINT is mentioned today, it will automatically produce the expectation
that a large proportion of content is source through the internet. Another
important author in this field, Stewart K. Bertram, explains:
\begin{displayquote}
older OSINT research was limited by both the coverage of its information and the
ability of the researcher to focus the capability on a speficic subject, be it
a person, location or topic. [\ldots] What has changed this status quo is the
arrival of the Internet, and particularly the explosion in the use of social
media technology circa 2000. The rise of these two technologies created a
multilingual, geographycally distributed, completely unregulated publishing
platform to which any user could also become an author and a publisher. [\ldots]
By increasing the coverage and focus of OSINT the Internet effectively promoted
OSINT from a supporting role to finally sit alongside other more clandestine and
less accessible investigative capabilities \cite{SB15}.
\end{displayquote}

These explosion of sources of information causes another problem: reliability.
Unless we are sourcing information for a scientifical magazine which
follows the rigorous fact checking protocol of most scientific researches, then
we are facing a vast landscape of information with various degrees of
veridicity: reliable fact-checked sources on one side and \emph{fake news} at
the other end of the spectrum. The work of the OSINT investigator is to move in
this virtually infinite universe of news, pick only relevant information and
establish how reliable they are. Again this is not new, it is called
\emph{intelligence analysis}: ``[\ldots] the application of individual and
collective cognitive methods to weigh data and test hypoteses within a secret
socio-cultural context'' \cite{JH07}. Information
are scored from A (reliable) down to E (unreliable), plus F (reliability
unknown).

Within the domain of counterfeit investigations, OSINT can have a number of
roles to play. First off and foremost, should be used to establish if the items
being sold are genuine or counterfeits. Secondly, it can be used to establish
the extent of profits made (and therefore loss of income for the original brand)
by the merchant selling them. And finally, OSINT can provide vital help in
identifying the identity of the merchant as well as that of the manifacturer.


\section{GDPR}
The General Data Protection Regulation (EU) 2016/679 (a.k.a. GDPR) is the latest
legislative effort of the Europea Union, that regulates the collection
and use of European citizens' data. The two main goals that guided this new
legislation were already set in 2015 by the Council of the European Union, which
stated: ``The twofold aim of the Regulation is to enhance data protection rights
of individuals and to improve business opportunities by facilitating the free
flow of personal data in the digital single market'' \cite{CoEU956515}.
Essentially the GDPR gives more control to European citizens over the use of
their data, thus restricting the ability of private and public organizations to
process Personally identifiable information (a.k.a. PII) data, and at the same
time it becomes a legislative framework that standardizes personal data
processing in every country member of the European Union. Let's see in more
detail what this entails and what are the consequences for a product like
Linkero.

First off it is important to clarify some of the terminology used in the
legislation. \emph{Personally identifiable information}, or PII, is defined in
privacy laws as one single piece of data that alone can identify a single person. For
instance first and last name, an email address, a physical address, a phone
number, passport serial number, credit card number, a picture portraying an
individual, a combiantion of username and password. GDPR applys to the
collection, process and storage of PII. A \emph{Data Controller} is any
organization that collects and manages PII from European citizens. That includes
online private companies (e.g. Google, Facebook the most obvious), as well as
public institutions like primary and secondary schools and hospitals to name
few. Another party that is involved in the handling of private data is the
\emph{Data Processor}, which in some cases it conicides the the Data Controller
itself, when data is analysed internally, or with any other organization (e.g.
contractors, vendors, suppiers, etc.) that receives PII from the Data Controller
and processes it in its behalf. Finally the \emph{Data Protection Officer} is a
public office that is appointed to overlook and audit every Data Controller in
its area of control.

What GDPR means for European citizens is that as of May 25\textsuperscript{th}
2018, when it officially came into force, it grants much more rights over their
own personal data, compared to previous legislations. These rights include
\emph{data portability}, which is to say that any Data Controller has to
structure the collection and storage of PII in such a way that those information
can be removed, transferred and re-allocated to another Data Controller without
any further modification. In other words, this is a way to guarantee
interoperability between Data Controllers, preventing single Controllers to
lock-in their users. GDPR also states that EU citizens have the \emph{right to
information and transparency}, so that at any moment they can request their PII
to be disclosed as it has been collected up to that point in time. Another right
that generated a lot of discussion, is the \emph{right to be forgotten}, which
states that users can have their information permanently erased from a Data
Controller, if they no longer require their services.

GDPR also set more stringent constraints to the operation of organization
that fall within its scope. GDPR is self-declared \emph{transnational in scope},
meaning that it applys to every company, anywhere in the world, as long as they hold EU
citizens' PII data. Obviously, enforcement is an issue in cases such as
companies that are based somewhere in the Cayman Islands. However this aspect is
mainly ment at large multinational corporations, that could simply
transfer their databases to another jurisdiction outside the EU. Those
multinational would still be subjected to EU law and would still be
accountable as long as they have legal presence in any EU member state,
regardless of where the PII data actually sits. GDPR also establishes a stronger
\emph{Data Subject consent}. This means that Data Controllers can no longer
assume that their users data can be processed and used for whatever purpose,
just because PII were given to them during the registration process, for
instance, nor they can pre-tick the data consent acknoledgment box, or ask the
user to tick a box only if they do not consent to the treatment of their data.
On the contrary, Data Controllers have to explicitly ask and receive users
consent, and consent has to be renewed over time. Finally, GDPR requires
organization to \emph{report data breaches} within 72 hours after the breach was
discovered. Data Controllers not only have to notify the supervisory authority
and their users, but the notification has to carry adequate details to explain
how the incident occurred, how many users are affected, and how is the
organization going to respond.

With all these new rights and obligations in place, the last part of the
legislation concernes relative and proportionate \emph{sanctions} for
organizations that fail to comply. There are three levels of penalties that can
be imposed:
\begin{enumerate}
  \item a warning letter for the first time an organization is found in breach
  of GDPR, and if the breach is non-intentional;
  \item a more serious breach could result in the order to commit to regular
  periodic data protection audits;
  \item and finally, if the breach is deemed to be serious enough, GDPR the
  prescribes two sets of financial fines depending of which obligations were
  missed by the Data Controller, the harsher of which could amount to up to
  \euro20 millions or up to 4\% of the annual turnover estimated from the
  prior financial year.
\end{enumerate}

The question now is: how does GDPR affect online services that for their
nature collect and process bits of PII that are publicly available? First and
foremost, GDPR applys only to PII belonging to EU citizens, therefore we can say
that an online service that collects PII can keep on providing its functions
without nay further thought as long as it stays clear of EU citizens, as cynical
as it may sound. But let's say some users will be pointing the service to PII of
European users, will than that be sufficient to be found in breach of GDPR? The
answer to this latter question is not a clear-cut \emph{no}, there are actually
use-cases where collection can still be an option. Obviously those cases do not
include when the Data Subject officially expressed consent to the organization
scraping his or her PII, that is not how data scraping works, especially in the
context of open source investigations. Although this may be a theoretical
argument, organizations may have a \emph{legitimate interest} in scraping,
storing and processing publicly available PII. In the specific context in which
Linkero is designed, we are talking about an Intellectual Property owner, or an
agent in their behalf, that has a vested interest in gathering information about
an online shop that sells their products, under market value, or without being
an officially approved reseller. Having said that, GDPR definitely limits and
restricts, for better or worse, the ability to extract, store and process public
PII.

To conclude this section, it is interesting to mention how ensuring
adequate privacy protection is a difficult balance and has its own trade-offs. This is
the case of the WHOIS protocol maintained by ICANN (i.e. Internet Corporation
for Assigned Names and Numbers), where internet domain registrants' PII are
kept and made public. With the introduction of GDPR, personal details of the
person who registered a domain will have to be hidden. Details of the domain
registrant are valuable information in order to identify the identity, or the
signature of the person or group behind specific internet domains. This is
particularly useful when security researchers try to attribute the ownership of
phshing domains to a particular actor or group. Likewise, in counterfeit
investigations, being able to tell how many and which domains were registered by
the same actor is particularly useful, especially if you are trying to monitor
the actor's activities. Information do not need to be accurate, in fact domain
registrars do not verify the real identity of the person registering a domain.
There are details that need to be valid, like contact details (e.g. email
address, phone number, \ldots) because on them depends the correct funcioning of
the domain, which is in the registrant's interest. The disappearence of those
PII from WHOIS cause a great deal of frustration in the security community, and
ended up with ICANN filing an official litigation against the European Union
asking for some sort of exhemption in order to keep offering relevant WHOIS
data.

GDPR was created and voted by EU member states in a period when information is
bein considered the new oil of the global economy: access to personal data and
consumer behaviour generates huge amounts of revenues for corporations like
Facebook and Google. And the reason for that is simple: targeted ads campaigns
are extremely effective and efficient, and marketing departments are willing to pay
for that. The commercial application of petabyte of consumer data is well known.
But there is another aspect to an unregulated access to data belonging to
billions of people, which is \emph{political influence}. As we have seen during
the 2016 US election, bad actors managed to use personal data of millions of US
citizens to both target specifically voters in swing states, and to analyze
general voters' opinion in order to design very effective slogans. The same
actors fabricated made up stories---literally, fake news---that would appeal to
the most visceral human fears in order to influence undecided voters. That
happended also during the British referendum of Brexit, and was attempted during
the French presidential elections in 2017, and very likely in many other
countries that did not receive as much media attention. At the same time we
saw government bodies abusing their ability to tap into and collect
indiscriminately data, just as Edward Snowden exposed how the US National
Security Agency. GDPR may not be the best solution yet, but it is a much needed
regulation at a time when personal information can have serious impact on the
society.

\section{Server security standards}
Loren Ipsum \ldots

\section{Ergonomy of software interface}
Loren Ipsum \ldots

\section{Code development principles}
Loren Ipsum \ldots

\section{Existing brand monitoring services}
Loren Ipsum \ldots