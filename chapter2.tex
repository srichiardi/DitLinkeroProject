\chapter{Literature review}

\section{Introduction}
This chapter will provide an overview of different topics that informed design
decisions made during the initial stages of the project. We will start looking
at the discipline of Open Source Intelligence, which refers to all those
protocols and techniques used by government and private agencies to piece
together intelligence reports using publicly available sources, since Linkero is
a tool that facilitates the structured collection and analysis of a limited
portion of open source data. The legal aspect of data collection and analysis
will be explored with a reasoned summary of what is GDPR and how it impacts
similar online services. Three sections will be dedicated to industry level best
practices in relation to the security of servers facing the public internet, the
usability of software interfaces and current guidelines to build reusable,
mantainable and expandable code. Finally we will review briefly what are some of
the current services already offering brand monitoring tools and how they differ
from one another.


\section{Open Source Intelligence}
Get books from work!

\section{GDPR}
Loren Ipsum \ldots

\section{Server security standards}
Loren Ipsum \ldots

\section{Ergonomy of software interface}
Loren Ipsum \ldots

\section{Code development principles}
Loren Ipsum \ldots

\section{Existing brand monitoring services}
Loren Ipsum \ldots