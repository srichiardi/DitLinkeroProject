\chapter{Literature review}

\section{Introduction}
This chapter will provide an overview of different topics that informed design
decisions made during the initial stages of the project. I will start looking
at the discipline of Open Source Intelligence, which refers to all those
protocols and techniques used by government and private agencies to piece
together intelligence reports using publicly available sources, since Linkero is
a tool that facilitates the structured collection and analysis of a selected
portion of open source data. The legal aspect of data collection and analysis
will be explored with a reasoned summary of what is GDPR and how it impacts
similar online services. Three sections will be dedicated to industry level best
practices in relation to the security of servers facing the public internet, the
usability of software interfaces and current guidelines to build reusable,
maintainable and expandable code. Finally I will review briefly what are some of
the current services already offering brand monitoring tools and how they differ
from one another.


\section{Open Source Intelligence}
Open Source Intelligence (more commonly referred to as OSINT) is a relatively
young discipline, that is concerned with the art of piecing together strategic
intelligence from public sources of information. Michael Bazzel, a leading OSINT expert,
defines it as:
\begin{displayquote}
any intelligence produced from publicly available information that is collected,
exploited, and disseminated in a timely manner to an appropriate audience for
the purpose of addressing a specific intelligence requirement. [\ldots] For the
CIA, it may mean information obtained from foreign news broadcasts. For an
attorney, it may mean data obtained from official government documents that are available to the
public. For most people, it is publicly available content obtained from the
internet \cite{MB15}.
\end{displayquote}

As Bezzell explains, OSINT is not necessarily based on online sources, at least
in its most broad definition. Journalistic style, old fashion dossiers filled
with newspaper clippings are a form of OSINT. However it is fair to say that
whenever OSINT is mentioned today, it will automatically produce the expectation
that a large proportion of content is sourced through the internet. Another
important author in this field, Stewart K. Bertram, explains:
\begin{displayquote}
older OSINT research was limited by both the coverage of its information and the
ability of the researcher to focus the capability on a specific subject, be it
a person, location or topic. [\ldots] What has changed this status-quo is the
arrival of the Internet, and particularly the explosion in the use of social
media technology circa 2000. The rise of these two technologies created a
multilingual, geographically distributed, completely unregulated publishing
platform to which any user could also become an author and a publisher. [\ldots]
By increasing the coverage and focus of OSINT the Internet effectively promoted
OSINT from a supporting role to finally sit alongside other more clandestine and
less accessible investigative capabilities \cite{SB15}.
\end{displayquote}

This explosion of sources of information causes another problem: reliability.
Unless we are sourcing information from a scientific magazine which
follows the rigorous fact checking protocol of most scientific researches, then
we are facing a vast landscape of information with various degrees of
veracity: reliable fact-checked sources on one side and \emph{fake news} at
the other end of the spectrum. The work of the OSINT investigator is to move in
this virtually infinite universe of news, pick only relevant information and
establish how reliable they are. Again this is not new, it is called
\emph{intelligence analysis}: ``[\ldots] the application of individual and
collective cognitive methods to weigh data and test hypothesis within a secret
socio-cultural context'' \cite{JH07}. Information
are scored from A (reliable) down to E (unreliable), plus F (reliability
unknown).

Within the domain of counterfeit investigations, OSINT can have a number of
roles to play. First off and foremost, should be used to establish if the items
being sold are genuine or counterfeits. Secondly, it can be used to establish
the extent of profits made (and therefore loss of income for the original brand)
by the merchant selling them. And finally, OSINT can provide vital help in
identifying the identity of the merchant as well as that of the manufacturer.


\section{GDPR}
The General Data Protection Regulation (EU) 2016/679 (a.k.a. GDPR) is the latest
legislative effort of the European Union, that regulates the collection
and use of European citizens' data. The two main goals that guided this new
legislation were already set in 2015 by the Council of the European Union, which
stated:
\begin{displayquote}
The twofold aim of the Regulation is to enhance data
protection rights of individuals and to improve business opportunities by facilitating the free
flow of personal data in the digital single market \cite{CoEU956515}.
\end{displayquote}
Essentially the GDPR gives more control to European citizens over the use of
their data, thus restricting the ability of private and public organizations to
process Personally Identifiable Information (a.k.a. PII) data, and at the same
time it becomes a legislative framework that standardizes personal data
processing in every country member of the European Union \cite{EU18}. Let's see
in more detail what this entails and what are the consequences for a product like
Linkero.

First off it is important to clarify some of the terminology used in the
legislation. \emph{Personally identifiable information}, or PII, is defined in
privacy laws as one single piece of data that alone can identify a single person. For
instance: the first and last name of a person, an email address, a physical
address, a phone number, passport serial number, credit card number, a picture portraying an
individual, a combination of username and password. GDPR applies to the
collection, processing and storage of PII. A \emph{Data Controller} is any
organization that collects and manages PII belonging to European citizens. That
includes online private companies (e.g. Google, Facebook the most obvious), as well as
public institutions like primary and secondary schools and hospitals to name
few. Another party that is involved in the handling of private data is the
\emph{Data Processor}, which in some cases coincides with the the Data
Controller itself, when data is analyzed internally, or with any other organization (e.g.
contractors, vendors, suppliers, etc.) that receives PII from the Data Controller
and processes it in its behalf. Finally the \emph{Data Protection Officer} is a
public office that is appointed to overlook and audit every Data Controller in
its area of control \cite{CR17}.

What GDPR means for European citizens is that as of May 25\textsuperscript{th}
2018, when it officially came into force, it grants much more rights over their
own personal data, compared to previous legislation. These rights include
\emph{data portability}, which is to say that any Data Controller has to
structure the collection and storage of PII in such a way that those information
can be removed, transferred and re-allocated to another Data Controller without
any further modification. In other words, this is a way to guarantee
interoperability between Data Controllers, preventing single Controllers to
lock-in their users. GDPR also states that EU citizens have the \emph{right to
information and transparency}, so that at any moment they can request their PII
to be disclosed as they have been collected up to that point in time. Another
right that generated a lot of discussion, is the \emph{right to be forgotten}, which
states that users can have their information permanently erased from a Data
Controller, if they no longer require their services \cite{EU18}.

GDPR also sets more stringent constraints to the operation of organizations
that fall within its scope. GDPR is self-declared \emph{transnational in scope},
meaning that it applies to every company, anywhere in the world, as long as they hold EU
citizens' PII data. Obviously, enforcement is an issue in cases such as
companies that are based somewhere in the Cayman Islands. However this aspect is
mainly meant at large multinational corporations, that could simply
transfer their databases to another jurisdiction outside the EU in an attempt
to circumvent its reach. Under the current definition of GDPR, those
multinationals would still be subjected to EU law and would still be
accountable as long as they have legal presence in any EU member state,
regardless of where the PII data actually sits. GDPR also establishes a stronger
\emph{Data Subject consent}. This means that Data Controllers can no longer
assume that their users' data can be processed and used for whatever purpose,
just because PII were given to them during the registration process. Nor they
can pre-tick the data consent acknowledgment box, or ask the  user to tick a box
only if they do not consent to the treatment of their data. On the contrary,
Data Controllers have to explicitly ask and receive users' consent, and consent
has to be renewed over time. Finally, GDPR requires organization to \emph{report
data breaches} within 72 hours after the breach was discovered. Data Controllers
not only have to notify the supervisory authority and their users, but the
notification has to carry adequate details to explain  how the incident
occurred, how many users are affected, and how is the organization going to
respond \cite{CR17}.

With all these new rights and obligations in place, the last part of the
legislation concerns to the \emph{sanctions} for organizations that fail to
comply. There are three levels of penalties that can be imposed:
\begin{enumerate}
  \item a warning letter for the first time an organization is found in breach
  of GDPR, and if the breach is non-intentional;
  \item a more serious breach could result in the order to commit to regular
  periodic data protection audits;
  \item and finally, if the breach is deemed to be serious enough, GDPR
  prescribes two sets of financial fines depending of which obligations were
  missed by the Data Controller, the harsher of which could amount to up to
  \euro20 millions or up to 4\% of the annual turnover estimated from the prior
  financial year, depending on which is higher \cite{EU18}.
\end{enumerate}

The question now is: how does GDPR affect online services that for their
nature collect and process bits of PII that are publicly available? First and
foremost, GDPR applies only to PII belonging to EU citizens, therefore we can say
that an online service that collects PII can keep on providing its functions
without any further thought as long as it stays clear of EU citizens. But let's
say some users will be pointing the service to PII of European users, will that
be sufficient to be found in breach of GDPR? The answer to this latter question
is not a clear-cut \emph{no}, there are actually use-cases where collection can
still be an option. Obviously those cases do not include Data Subjects
who expressed consent to their PII to be collected and stored by anyone online,
that is not how data scraping for intelligence purpose works, especially in the
context of open source investigations. Although it is too early for similar
cases to have been tried, GDPR accepts that organizations may have a
\emph{legitimate interest} in scraping, storing and processing publicly
available PII. The protection of intellectual property is clearly
mentioned as one of the rights that fall within the boundaries of legitimate
interest \cite{EU18}. In the specific context in which Linkero is designed, we
are talking about an Intellectual Property owner, or an agent in their behalf,
that has a vested interest in gathering information about an online shop that
sells their products, under market value, or without being an officially
approved re-seller. Having said that, GDPR definitely limits and restricts, for
better or worse, the ability to extract, store and process public PII.

To conclude this section, it is interesting to mention how ensuring
adequate privacy protection is a difficult balance and has its own trade-offs. This is
the case of the WHOIS protocol maintained by ICANN (i.e. Internet Corporation
for Assigned Names and Numbers), where internet domain registrants' PII are
kept and made public. With the introduction of GDPR, personal details of the
person who registered a domain will have to be hidden. Details of the domain
registrant are valuable information in order to identify the person, or the
signature of the person or group behind specific internet domains. This is
particularly useful when security researchers try to attribute the ownership of
phishing or spamming domains to a particular actor or group. Likewise, in
counterfeit investigations, being able to tell how many and which domains were registered by
the same actor is particularly useful, especially if you are trying to monitor
the actor's activities. Information do not need to be accurate, in fact domain
registrars do not verify the real identity of the person registering a domain.
There are however details that need to be valid, like contact details (e.g.
email address, phone number, \ldots) because on them depends the correct functioning of
the domain, which is in the registrant's interest. The disappearance of those
PII from WHOIS caused a great deal of frustration in the security community, and
ended up with ICANN filing an official litigation against the European Union
asking for some sort of exemption in order to keep offering relevant WHOIS
data.

GDPR was created and voted by EU member states in a period when information is
being considered the new oil of the global economy: access to personal data and
consumer behaviour generates huge amounts of revenues for corporations like
Facebook and Google. And the reason for that is simple: targeted ads campaigns
are extremely effective and efficient, and marketing departments are willing to pay
for that. The commercial application of petabyte of consumer data is well known.
But there is another aspect to an unregulated access to data belonging to
billions of people, which is \emph{political influence}. As we have seen during
the 2016 US election, bad actors managed to use personal data of millions of US
citizens to both target specifically voters in swing states, and to analyze
general voters' opinion in order to design very effective slogans. The same
actors fabricated made up stories---literally, fake news---that would appeal to
the most visceral human fears in order to influence undecided voters. That
happened also during the British referendum of Brexit, and was attempted during
the French presidential elections in 2017, and very likely in many other
countries that did not receive as much media attention. At the same time we
saw government bodies abusing their ability to tap into and collect
indiscriminately data, just as Edward Snowden exposed how the US National
Security Agency. GDPR may not be the best solution yet, but it is a much needed
regulation at a time when personal information can have serious impact on the
society.

\section{Server security standards}
A chain is as strong as its weakest link, and a system is as secure as its
most vulnerable element, as the saying goes. In today's reality, any system
facing the open and public internet has to be secured against people trying to
misuse it. It is not just big banks and multinational corporations that have to
put in place sci-fi grade defenses, but the same private wifi routers and
connected thermostats have to be protected against bad actors that would gladly
take them over and deploy them to perform fraudulent or criminal
activities. It is well known that professional computer criminals constantly
scan the internet to detect new connected devices and try to add them to their
personal collection of bots for different purposes.
Internet security is therefore a must regardless of the complexity of the
system. Information Security literature identifies three main goals of a good
system security strategy, also referred to as the \emph{CIA triad} \cite{DP00}:
Confidentiality, Integrity and Availability. That means that a system security
strategy aims at guaranteeing that data stored in the system will be accessed
only by legitimate and relevant parties (confidentiality), that the same data
will be kept in its original state away from sources that may alter,
intercept or erase it (integrity), and that despite of the safety measures in
place, if an attack succeeds, the data will still be accessible, and business
will continue with minimal to no disruption (availability). In addition to the
CIA triad, some experts also include \emph{accountability}, which essentially
means that if an incident occurs, the remediation team should be able to
identify exactly where the breach happened and the location if not the identity
of the person who executed the attack.

The industry identifies two main types of threat: internal and external.
The external threat is probably the first and most popular to come to mind,
often depicted like the hooded boy wearing a Guy Fawkes mask and typing away on
his laptop: the hacker. Put simply, it is a person or organization that attacks
a system from the outside. The other threat that does not get as much
attention, but can be equally damaging, is the internal threat. The internal
threat is represented by an internal user that intentionally or by mistake
causes damage to the system. An attacker can be further classified based on
his/her level of activity: a \emph{passive} attacker tend to sit and intercept
data exchanges, without affecting the system resources, whereas an \emph{active}
attacker directly alters system resources while the operation is underway
\cite{WS15}.

Any attack normally impacts one or more of system assets: hardware, software,
data or networks. There are five types of security
measures that an organization can put in place to protect its assets:
deterrent, prevention, detection, response and recovery. Deterrent measures are
meant to scare attackers off before they even consider to start their exploits.
The criminal law with its system of penalties is definitely the first deterrent
that comes to mind. But media articles detailing how sophisticate and complex is
a detection system can offer the same purpose. Preventive measures on the other
end are technologies that anticipate potential attacks and stop them before any
damage occurs. Firewalls are a great example. Since absolute prevention
is rarely feasible, detection technologies can cover those situation in
which an attack is found while still ongoing. Detection immediately
triggers response measures, such as blacklisting an IP associated with a
number of failed login attempts. Finally, recovery measures are the ones that
are deployed after an attack occurred, and try to assess the damage and
remediate where possible \cite{BL04}.

Information security is a very vast and complex discipline, which
directly reflects the complexity of modern interconnected systems. For the
purpose of this project however I will only consider the basic security
configuration required to protect a standalone Linux server.

In the next part of this section I will review in greater details some of the
prevention technologies and solutions available, particularly those that were
considered for this project, focusing on prevention, detection and
recovery measures.

\subsection{Prevention}

\subsubsection{Regular software updates}
Applications running on the system are rarely bug-free. It is always good
practice to keep every system application up-to-date with latest patches, since
every time a new vulnerability is discovered, application developers tend
to issue a new version or patch to close it \cite{WS15}. This system is set to
look for security updates issued by the Linux distribution Debian, every 24
hours.

\subsubsection{Environment isolation}
Rather than a technology, \emph{isolation} is a principle and a policy that
recommends every resource should be isolated from the rest, and access between
them should be regulated and monitored. This is particularly critical in the
context of systems that have public interfaces. Isolation applies to file
management as well, requiring users files and folders to be isolated from one
another, preventing one user's compromised credentials to be used to access
everyone else's files. Finally, isolation applies to the same security
mechanisms, since they should be isolated from the rest of the system, so that
they cannot be compromised by the attacker \cite{WS15}. For this project,
isolation was implemented by giving each application (e.g. web server, email
server, databases, etc\ldots) its individual user profile and access, so that if
one is compromised, the indruder will still be limited to the set of tasks
granted to that application.

\subsubsection{Virtual Private Networks}
VPN are a good solution to connect parts of the system that are not located on
the same machine, or in the same network. Since communication between resources
has to go through the public network, VPN offer a way to encrypt and protect
data exchange based on asymmetric encryption technology \cite{WS15}. Currently
there are no VPN deployed in this project, since every element is installed
on the same single server. However VPN would be the ideal solution, if the
system grows and specific applications, like databases, would have their
dedicated machines.

\subsubsection{SSL and TLS encryption}
Similarly to VPN, SSL and TLS are two technologies that create an encrypted
communication channel between a client and a server. The most common use case
for these solutions is when login forms are required, so that a client can
transfer login credentials to the server using a HTTPS connection, thus avoiding
to transfer username and password in clear over a non-secure public network
\cite{WS15}. For this project, an SSL certificate was issued for free by
\emph{Let's encrypt}\texttrademark (\url{https://letsencrypt.org}).

\subsubsection{SSH}
The Secure Shell, SSH, is the default application for remote server management.
Like VPN and SSL, it establishes an encrypted connection between a client and a
server, relying on asymmetric encryption. It allows system administrators to
login remotely to the server command line shell. It can be configured to either
accept a standard password login, or to accept automatic logins from approved
clients only \cite{WS15}. Every admin interaction on this system happens through
SSH tunnel.

\subsubsection{Firewalls}
Firewalls can be described as \emph{intrusion prevention systems}. A firewall is
an application that filters all incoming and outgoing network traffic between
the internal network and the public internet, and responds to a set of rules to
decide was is allowed and what is not. It provides a single choke point
for network traffic, where the system can act as detection mechanism for
suspicious connections, as well as traffic logger for audits. There are four
different types and corresponding level of complexity: packet filtering,
stateful inspection, application-level gateway and circuit-level gateway. The
last two are the most sophisticated and require also a lot of CPU and memory
resources, for this reasons are not suitable for this project \cite{WC03}.

Packet filtering is the most basic capability of a firewall, and bases its
decision on whether to allow or reject a connection on the information contained in an IP
packet: source and destination IP, source and destination transport-level
address, IP protocol and interface. Packet filter firewalls however are
incapable of establishing when a connection to a port number above 1024 is
legitimate or not. This is where a stateful inspection firewall comes into play.
Stateful packet inspection works by keeping track of ongoing connections,
session numbers, and originating party in order to prevent attacks like session
hijacking \cite{WS15}. I will discuss the details of the system firewall in the
next chapter.

\subsubsection{Users and groups least privileges}
Just like environment isolation, assigning users and groups with the right
amount of privileges and permission is more a policy than a technology. This
policy states that users should be given the minimum amount necessary of
privileges in order to perform their tasks. This way if an account is
compromised, there is a chance that damages can be controlled and limited.
Permissions include the ability to read, write, execute, delete or share a
specific resource. More advanced resource and permission management policies
include a temporary aspect, in that permissions are granted for a limited amount
of time, and are active only during working hours, or whenever a specific task
needs to be performed \cite{WS15}.

\subsubsection{Password policy}
A password policy is design to make sure that all users comply with the industry
best known standards of password security. A password policy should
establish that:
\begin{itemize}
  \item the strength of the password by insuring and checking that a user
  generated password uses the most varied combination of lower and upper case
  letters, numbers and special symbols in a string of a minimum length of
  eight characters, in order to guarantee that a brute-force attack (attempting
  to guess the password trying every possible combination) would require a
  significant amount of time to succeed;
  \item the lifetime of a password, which should be long enough not to cause
  user friction, but short enough to guarantee that should the password be
  compromised, there is a chance it will be too late to use it before
  expiration;
  \item a new password cannot be the same as any previously used;
  \item a set of alternative contact details and security questions that should
  be used to recover a lost or compromised password;
  \item a maximum numbers of failed login attempts, after which the account is
  locked and a password reset is required;
  \item recommendations about the right way of storing the password: making sure
  the password is not written down or saved in an unsecured file.
\end{itemize}

\subsubsection{Encryption of data at rest}
Systems that store data in databases may be compromised, when that happens if
the data is stored in clear, the attacker can make a copy of the database, and
access all the contents. Data that is not loaded in memory can be encrypted, so
that simply making a copy and moving it outside of the current system, will not
allow the attacker to access it, lacking the necessary decryption keys
\cite{WS15}.

\subsubsection{Physical security}
Physical and infrastructure security is sometimes an overlooked aspect, when the
focus is all on hardening the software and data, when it is actually as
important. There are three main categories of threats to physical
infrastructure: environmental, technical and human caused. Environmental threats
includes both exceptional events like tornadoes, floods or earthquakes, as well
as less traumatic events like the level of humidity, dust, external temperature, gas
and other chemical elements, factors that in the long run can actually cause
serious damage. Technical threats include power surges or drops, and
electromagnetic interference. Finally the most common human-caused threats are:
unauthorized physical access, vandalism, theft and misuse.

For the purpose of this project however, no countermeasures were taken to
mitigate those threats, since the system is composed of a single virtual server,
run by the Virtual Private Server provider Linode ltd. I assume that safety and
security measures are being adopted by the service provider \cite{WS15}.

\subsection{Detection}

\subsubsection{Intrusion detection systems}
Intrusion detection system are the second level of defence, in case all
of the previous preventive measure fail to stop an external intruder. The basic
idea behind an intrusion detection system (IDS) is that the way an intruder
interacts with the system is substantially different from any other authorized
user, thus his/her behaviours can be described in quantitative terms (the
intruder's signature) and this description can then be implemented in a system
whose purpose is to ring an alarm when such behaviour is detected and block the
intruder before any damage occurs \cite{WS15}.

These are the main types of IDS currently available:
\begin{description}
  \item[Host-based IDS]: application that runs on a specific host, and only
  monitors activity on that host.
  \item[Network-based IDS]: application that runs on a host part of a larger
  network, that monitors all the activities within that network.
  \item[Distributed or hybrid IDS]: application that gathers data from a
  cluster of sensors distributed across a network \cite{RS00}.
\end{description}

There are essentially two ways that an IDS can be programmed to detect
intruders: we can either decide that a range of behaviours is legitimate,
therefore everything else that deviate from that norm is an intruder, or we can
list bad behaviours, and tell the IDS to trigger the alarm when any of the
behaviours listed is detected. The first strategy is called \emph{anomaly
detection} the latter \emph{signature detection}. Without going in too many
details, anomaly detection is clearly the most effective of the two, and would
be anyone's first choice if not for the fact that is very difficult to define a
norm. So difficult that it requires very sophisticate approaches like
statistical analysis and machine learning techniques \cite{GT09}. Signature
detection on the other hand is a bit easier to implement, and guarantees a lower
number of false positives, given that its database is stored with good quality
intruders' signatures. The problem is that any new, unexpected and undefined
intrusion behaviour will not be detected, leaving the system vulnerable to
zero-day attacks \cite{WS15}.

\subsubsection{Regular file, service and vulnerability audits}
A security auditing system is a very complex set of hardware and software tools,
policy, processes and old fashion human labour. Such a system is beyond the scope of
this project, but it is worth briefly mentioning it, for completeness sake,
within the discussion on information security.

A well designed security auditing system has the following requirements:
\begin{itemize}
  \item being able to establish at any time whether a system's security is
  intact or has been breached;
  \item capture by default all data that is necessary to establish if an
  intrusion occurred, when, how and possibly by who;
  \item store relevant data for future analysis and forensics;
  \item feed anomaly signatures back to an intrusion detection system signature
  database \cite{WS15}.
\end{itemize}

An audit can be triggered right after a potential attack is detected as well as
can be repeated periodically. An audit can be triggered in different ways:
\begin{description}
  \item [basic alerting]: is the simplest for of trigger, which only requires
  a single rule to trigger an alarm if a specific event occurs.
  \item [Baselining]: is the practice of defining a normal versus unusual
  event or pattern. This feature overlaps with an IDS signature-based trigger.
  \item [Correlation]: is taken from statistical analysis and consists on
  triggering an alarm or notification when an event or set of events is used to
  infer the presence of a potential breach \cite{WS15}.
\end{description}

\subsection{Recovery}

\subsubsection{Business continuity plan}
A business continuity plan is a set of measures, mechanisms that ensure an
organization can go on or quickly restore basic and critical operations
with minimal downtime \cite{TW10}. In a business continuity plan a series of
\emph{disaster} scenarios are identified, for instance: loss of workspace, loss
of personnel, loss of network infrastructure. Every scenario is coupled with a list of
function critical resources that can be impacted and relative countermeasures
to be executed along with roles and responsibilities assigned to
specific people. Having a backup of all the data on the system
is a preventive measure, using it to restore data that was lost, encrypted or
corrupted during an attack is what recovery looks like.

\section{Design Engineering}
Goal of a good software project is to produce software that not only works, but
that is easy to understand, to test, maintain and expand if needed.
Helwett-Packard developed five attributes \cite{RG87} that define the quality of
a software product:
\begin{description}
\item [Functionality] concerns the set of functions and capabilities of the
program, as well as its security.
\item [Usability] relates to how well the intended users are able to operate the
program.
\item [Reliability] is the measure of frequency and severity of the program
failures.
\item [Performance] indicates the program speed of response, resources
consumption, efficiency and throughput.
\item [Maintainability] is concerned with how the software is easy to
troubleshoot, adapt, configure and extend.
\end{description}

Design engineering takes care of the quality of a software, and is the last
stage of software modelling before code development can actually start
\cite{RP05}.
Design engineering is divided in four elements: data design, component-level design,
interface design and architectural design. I will briefly explore each one
here.

\subsection{Component-level design}
Component design refers to the procedural aspect of a software program:
functions, processes and algorithms are the object of this element of design.
Design engineering developed over the years a set of rules and guidelines that
allow developers to write code that is maintainable and extendable:
\begin{description}
\item [Modularity] states that the final software should be the result of
separate nuclear components, called modules, that will be integrated and
interact with each other in order to produce the desired functionalities
\cite{GM78}.
\item [Information hiding] suggests that, as much as possible, the data used
within one module for its operations, should be hidden from other modules. Only
the bare minimum amount of information should be exchanged between modules to
ensure software operations \cite{DP72}.
\item [Functional independence] is a concept directly derived by the previous
two, and it says that a module should focus on a specific functionality
(measured in level of cohesion) and it should have as little interaction with
other modules as possible (expressed in terms of coupling) \cite{DP72}.
\item [Refinement] rather than a prescription this is a top-down process of
defining each element of the software starting from a high level of abstraction,
and progressively narrowing down to more detailed procedural levels \cite{NW71}.
\item [Refactoring] is another technique that aims at simplifying the design
of a program without changing its functionality. The code is examined for
unused elements, redundancy, inefficient or unnecessary algorithms \cite{RP05}.
\end{description}

Robert Martin took those rules and techniques and elaborated further into his
five S.O.L.I.D. principles \cite{RM03}:
\begin{description}
\item [Single responsibility] states that each module or class should be
responsible for one function, and that function should be well encapsulated
within the module or class. This principle combines the concept of cohesion and
information hiding.
\item [Open-Closed] dictates that a module or a class should be built in such a
way to prevent modification, but still offer an interface that allows it to be
inherited and expanded. The reason is that, assuming a new requirement arise
calling for a new functionality to be implemented, adding that functionality
to an established class could alter its behaviour and cause unintended
consequences in other parts of the code where that class is in use. Open-closed
principle suggests instead a new subclass should be created in order
to fulfill the new requirement, thus guaranteeing the stability of an
application, while allowing for functionality extension.
\item [Liskov substitution] is an extension of the \emph{open-closed} principle.
It requires that any child or grand-child class can take its parent's class
place in the application without altering the general functionality of the
application. This principle is intended to guarantee consistency of interface
within class types of the same family.
\item [Interface segregation] recommends to develop simple
client-specific interfaces, rather than one general purpose interface for every
known client. One general purpose interface would include all functionalities
that are relevant to any client, thus forcing every client to implement
functionalities that are not necessarily relevant. The other problem is that if
a new requirement calls for a new interface to be added, all pre-existing
clients should update their implementation to fit the new interface, whether
they support it or not. To solve this issue, single functionalities should have
their own interface, and clients can implement selectively just those that are
relevant.
\item [Dependency inversion] finally integrates the concept of module
de-coupling, stating that higher level modules should not depend on lower level
ones. The problem with modules dependencies is that a change in one could force
changes to its dependees, the assembly of modules could result is expensive
operations to guarantee that every dependency is respected. This goes against
module re-usability. The solution is to add a standard interface between two
modules. A good example to explain this principle is imagining the relation
between a lamp and the home electrical supply: making them directly dependent
equates to soldering the lamp wires to the electrical system, instead the
electrical socket was developed and became the standard interface between any
electrical tool and the electrical power supply \cite{RP05}.
\end{description}

\subsection{User interface design}
User interface design is concerned with making sure the user will be able to
actually use the software, by designing an interface that is meaningful and
intuitive. Theo Mandel developed three \emph{golden rules} \cite{TM97}:
\begin{description}
\item[Place the user in control] means that the interface has to be designed
with the user's needs in mind, not the developer. And these are some of the
most common needs from a user perspective: a user needs to be able to perform
the same action in different ways (e.g. via console commands, mouse action,
keyboard short cuts, etc.); a user sometime needs to suspend a task and move to
a new one, before resuming the old one, as well as being able to \emph{undo} an
action made by mistake; a user needs adjust its interaction according to his/her
level of expertise (e.g. initial point and click can be replaced by macro
recording of a sequence of actions); a user does not need to see nor interact
with low level functions; a user needs to manipulate objects of a visual
interface as he/she would do if those objects were a physical thing.
\item[Reduce the user's memory load] requires for interfaces to be intuitive,
without much study a user should be able to know what each part does just by
looking at it. Mandel \cite{TM97} further specify this rule, by saying that the
visual layout should be modelled based on a real-wold metaphor using well
established visual cues; the initial default configuration should make sense for
the average user, allowing the expert user to change it to his/her liking;
information should be presented in a progressive way, starting from general
task, down to more detailed functions.
\item[Make interface consistent] demands that a standard layout should be
defined and kept consistent throughout the rest of the program. Context layers
are very important, allowing the user to know at a glance where his/her current
task lays within the general process (e.g. window tiles, graphical icons, color
coding, etc.), how the user was able to access the current interface and what
are the available next steps. Interface consistency also requires that design
takes interactive sequences that have become standard (e.g. \emph{Cntrl-C} is
the standard for copying pieces of text) and makes them available in the
application.
\end{description}

\subsection{Software architecture}
Software architecture encompasses two elements of software design: data design
and architectural design. Here I will touch on the main concepts that are used
to describe data and architectural design.

\subsubsection{Data design}
Just like this project, most recent software applications and services have been
developed around the idea that accessing, storing and computing large amounts of
data can offer innovative solutions. Which is a way of saying that data today is
the new black gold, it is not a surprise than that designing what kind of data
will be analyzed and how it will be stored and structured is extremely crucial
\cite{RP05}.

At architectural level, data design is mainly concerned with the structures that
will contain the data and how those integrate within the overall architecture.
We are talking about using external databases to store the business data.
Depending on the size and complexity of data requirements, we may see a small
amount of centralized databases that store production data that is needed to
provide live core business services. With bigger corporations and a growing
number of diversity of database structures, it may become more difficult for
data analysts to access information that is spread across multiple databases
with multiple diverse data structures. \emph{Data warehouses} come to the
rescue. Data warehouses essentially store copy of production business data in a
separate environment, and make sure that all the data is kept using the same
structure, thus allowing data analysts to perform data mining on current and
historical data, in order to offer strategic insight for business
executive decisions \cite{RM96}.

\subsubsection{Architectural design}
Architectural design aims at building a blueprint of the entire system, with
special emphasis of how each component interacts with the others. There are two
aspects that describe an architecture: style and pattern \cite{RP05}.

Style refers to the shape of the overall system structure, it is a static
representation of the system: which components interface directly to
each other, which are peripheral, which are central, and so on. Examples of
architectural styles include:
\begin{description}
\item[Data-centered] architecture refers to a system that is designed around its
database, where all other components are clients that exchange information only
with the central server, with little or no interaction with each other.
\item[Data-flow] architecture instead is applied to a system where data moves
mainly in one direction, first as input, leaving the system as output, after all
the modules located in-between these two points have modified somehow the
original input.
\item[Call and return] is a style in which modularity is central to the overall
operation, where each function is broken down in a hierarchy of modules. This
style has the advantage of being relatively easy to modify and scale.
\item[Layered] architecture, finally, describes a system in which components are
grouped from an outer layer, normally the user interface, through an application
and utility layer, down to the core layer which performs machine levels
operations.
\end{description}
These are just a sample of the most common architectural styles \cite{RP05}.

An architectural pattern instead describes the most dynamic aspect of the
system: how each component interacts with the others, how they behave.
Architectural patterns are also referred to as a solution to a standard software
development problem. These are some examples of patterns and the problem they
solve:
\begin{description}
\item[Concurrency] is the ability to perform multiple, different and
independent tasks at the same time. One problem developer face is when the
system has limited resources, like one single CPU, so that real multitasking
cannot be achieved. An \emph{operating system process management} pattern
provides that concurrency can be simulated by assigning a turn to each task for
a limited amount of runtime. This solution offers a real advantage for tasks
that depend on many read and write operations with external slow components.
\item[Persistence] is the ability to store data, after the process that created
it is finished. \emph{Application level persistence} pattern for instance
provides that applications show generate and manage their own file format where
to store user-generated data or software configurations. Another pattern that
offers a solution is the \emph{database management system}.
\item[Distribution] addresses the issue of components communication in a
distributed environment, like clients exchanging emails: each client cannot
assume that the recipient will be online in order to receive the communication.
The \emph{broker} pattern's solution consists in building an agent to delegate
to a communication, and the agent will guarantee the message delivery by waiting
for the recipient to be ready to receive the message \cite{JB00}.
\end{description}

\section{Existing brand monitoring services}
There are several  online companies offering what they call \emph{brand
monitoring} services. This type of service has two major definitions:
\begin{itemize}
  \item on the one end, brand monitoring refers to activities whose purpose is
  to analyze the online presence of a brand to gain insight about its the market
  reception and placement \cite{techpd}.
  \item on the other end, it refers to activities whose aim is to protect the
  intellectual property of a brand by collecting information on online
  activities that may infringe it \cite{cscbrand}.
\end{itemize}

Of the first category there are different platforms offering marketing
analytics, promising to let you know what people think instantly: Brands
Eye\texttrademark\footnote{\url{https://www.brandseye.com}},
Brandwatch\texttrademark\footnote{\url{https://www.brandwatch.com}}, among the
most popular. These services tend to monitor mostly social media platforms,
where consumer sentiment can be gauged. On the other end of the spectrum
Watchdog\texttrademark\footnote{\url{https://watchdogapp.com}} is an example
of platform that is much closer to this project idea, whose tag line summarizes
its purpose: ``Watchdog is an online surveillance tool
designed from the ground up by investigators''\cite{wtchd}. They enable searches
across the major 50 e-commerce platforms, including Amazon and eBay.
Unfortunately it is hard to get a better grasp of the mechanics of the platform,
or even the UI, since the service is offered on subscription only.

