\chapter{Literature review}

\section{Introduction}
This chapter will provide an overview of different topics that informed design
decisions made during the initial stages of the project. We will start looking
at the discipline of Open Source Intelligence, which refers to all those
protocols and techniques used by government and private agencies to piece
together intelligence reports using publicly available sources, since Linkero is
a tool that facilitates the structured collection and analysis of a limited
portion of open source data. The legal aspect of data collection and analysis
will be explored with a reasoned summary of what is GDPR and how it impacts
similar online services. Three sections will be dedicated to industry level best
practices in relation to the security of servers facing the public internet, the
usability of software interfaces and current guidelines to build reusable,
mantainable and expandable code. Finally we will review briefly what are some of
the current services already offering brand monitoring tools and how they differ
from one another.


\section{Open Source Intelligence}
Open Source Intelligence (more commonly referred to as OSINT) is a relatively
young discipline, that is concerne with the art of piecing together strategic
intelligence from publich sources of information. Michael Bazzel, a leading OSINT expert,
defines it as:
\begin{displayquote}
any intelligence produced from publicly available information that is collected,
explited, and disseminated in a timely manner to an appropriate audience for the
purpose of addressing a specific intelligence requirement. for the CIA, it may
mean information obteined from foreign news broadcasts. For an attorney, it may
mean data obtained from official government documents that are available to the
public. For most people, it is publicly available content obtained from the
internet.
\end{displayquote}

As Bezzell explains, OSINT is not necessarialy based on online sources, at least
in its most broad definition. Journalistic style, old fashion dossiers filled
with newspaper clippings are a form of OSINT. However it is fair to say that
whenever OSINT is mentioned today, it will automatically produce the expectation
that a large proportion of content is source through the internet. Another
important author in this field, Stewart K. Bertram, explains:
\begin{displayquote}
older OSINT research was limited by both the coverage of its information and the
ability of the researcher to focus the capability on a speficic subject, be it
a person, location or topic. [\ldots] What has changed this status quo is the
arrival of the Internet, and particularly the explosion in the use of social
media technology circa 2000. The rise of these two technologies created a
multilingual, geographycally distributed, completely unregulated publishing
platform to which any user could also become an author and a publisher. [\ldots]
By increasing the coverage and focus of OSINT the Internet effectively promoted
OSINT from a supporting role to finally sit alongside other more clandestine and
less accessible investigative capabilities.
\end{displayquote}

These explosion of sources of information causes another problem: reliability.
Unless we are sourcing information for a scientifical magazine which
follows the rigorous fact checking protocol of most scientific researches, then
we are facing a vast landscape of information with various degrees of
veridicity: reliable fact-checked sources on one side and \emph{fake news} at
the other end of the spectrum. The work of the OSINT investigator is to move in
this virtually infinite universe of news, pick only relevant information and
establish how reliable they are. Again this is not new, it is called
\emph{intelligence analysis}: ``[\ldots] the application of individual and
collective cognitive methods to weigh data and test hypoteses within a secret
socio-cultural context'' (Hayes Joseph (2007), Analytic Culture in the U.S.
Intelligence Community, Ed. Center for the Study of Intelligence). Information
are scored from A (reliable) down to E (unreliable) plus F (reliability
unknown).

Within the domain of counterfeit investigations, OSINT can have a number of
roles to play. First off and foremost, should be used to establish if the items
being sold are genuine or counterfeits. Secondly, it can be used to establish
the extent of profits made (and therefore loss of income for the original brand)
by the merchant selling them. And finally, OSINT can provide vital help in
identifying the identity of the merchant as well as that of the manifacturer.


\section{GDPR}
Loren Ipsum \ldots

\section{Server security standards}
Loren Ipsum \ldots

\section{Ergonomy of software interface}
Loren Ipsum \ldots

\section{Code development principles}
Loren Ipsum \ldots

\section{Existing brand monitoring services}
Loren Ipsum \ldots