\chapter{Abstract}
In this project I aim to build a data aggregator service for investigations and
monitoring of counterfeit products online. The service will be implemented with
a web interface for clients' access, an http server and specialized databases
depending on data and performance requirements.

The core mission of this project is to try to simplify daily tasks of counterfeit
investigations, and leverage server technologies to allow investigators to see
the bigger picture painted when data from one case connect to one or multiple
other cases previously unconnected.

Equally important will be the ability to deliver a product that is ready for
production environment, that is: running on a virtual server accessible from the
internet, protected from unauthorized intrusions and easily scalable should the
user population grow significantly.

In this project I will also discuss pros and cons of data collection from online
sources: from open API offered by online services to the community of
developers, to unstructured data that can be scraped and schematized, and which
legal implications need to be taken into account in light of the new European
General Data Protection Regulation (aka GDPR).
