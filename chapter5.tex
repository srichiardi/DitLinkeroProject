\chapter{Evaluation}

\section{Lessons learned}
Configuring third party applications can be as daunting and time consuming than
developing your own code, and it is definitely more frustrating when for some
unknown reason individual configurations do not seem to work.

The Django framework is supposed to make web-developers' work quick and easy,
unless it is the first time you use it: steep learning curve.

Django's ORM is a convenient way to interact with a relational database, keeping
the code consistent and clean, as long as there are no advanced table joins
needed, in that case SQL is a much simpler language.

I anspired to apply the principles of the SOLID method, but that remained in
larg part an aspiration. I realized that, given the time constraints, having a
working system was the priority over clean and elegant coding practices.

Testing and breaking your own code is emotionally challenging, but a test-driven
development, such as the approach recommended in the Extreme Programming
methodology, is a way to circumvent this emotional attachment, and to design
applications based on what they should do as well as how they could fail, which
is an aspect that could otherwise be oversought.

Estimating in advance the time required to comple each part of the project has
been very difficult, especially when learning time needed to be factored in, the
best strategy seemed to be to put as much hours as early as possible, rather
than trying to distribute evenly the work throughout a period of 12 months.

\section{Future work}
Linkero is fully operational and is currently being used by 3 people on a weekly
basis. Nonetheless, in its current implementation, it is a limited platform,
especially for two principal aspects:
\begin{itemize}
  \item there is no text seach feature to retrieve historical data that is
  currently available in the local database;
  \item external data gathering ability is limited to eBay data.
\end{itemize}

The feature that should be implemented first is a search engine to take
advantage of MongoDB's text searching capabilities. The reason for this is that,
storing historical data only makes sense if it is possible to consume it
afterward. Once the data is extracted and stored from external sources, it is
possible to interrogate it in many different ways that are not always available
through third party API. For instance, extremely useful is to see if one string
(e.g.
an email address) is 

\section{Conclusions}

